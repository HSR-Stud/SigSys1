%%%%%%%%%%%%%%%%%%%%%%%%% 
% Dokumentinformationen %
%%%%%%%%%%%%%%%%%%%%%%%%%
\newcommand{\titleinfo}{Signale und Systeme 1 - Zusammenfassung}
\newcommand{\authorinfo}{Schmidt, Selvarajah, Braun \& Co, Rast \& Co., Gwerder, K\"orner, Badertscher, Bruhin, Ehrler, Leuenberger}
\newcommand{\versioninfo}{$Rev: 2.0 $ | gem\"ass Unterricht Heinz Mathis/HS2019} 
					
%%%%%%%%%%%%%%%%%%%%%%%%%%%%%%%%%%%%%%%%%%%%%
% Standard projektübergreifender Header für
% - Makros 
% - Farben
% - Mathematische Operatoren
%
% DORT NUR ERGÄNZEN, NICHTS LÖSCHEN
%%%%%%%%%%%%%%%%%%%%%%%%%%%%%%%%%%%%%%%%%%%%%
\include{header/header}
\fancyfoot[C]{
}
\usepackage{pdfpages, tabularx}
\usepackage{pgfplots}
\usepackage{tikz}

\newcommand{\beispiel}[1]{\textcolor{brown}{\mbox{#1}}}
\newcommand{\formel}[1]{\fcolorbox{black}{yellow}{#1}}
\iffalse
\newenvironment{graphPlot}[2][\unskip]{
	\begin{tikzpicture}[xscale=0.8, yscale=0.8]
		%\draw[help lines] (0,0) grid (6,4);
		\normalsize
		
		\draw [<->] (0, 2.5) -- (0, 0) -- (5, 0);
		\node [right] at (5, 0) {x};				% x-label
		\node [above] at (0, 2.5) {y(t)};			% y-label
		
		\draw (1, -0.1) --  (1, 0.1);				% Strich
		\node [below] at (1, 0) {$\frac{1}{2}$};	% Zahl beim Strich
		\draw (2, -0.1) --  (2, 0.1);
		\node [below] at (2, -0.1) {$1$};
		\draw (3, -0.1) --  (3, 0.1);
		\node [below] at (3, 0) {$\frac{3}{2}$};
		\draw (4, -0.1) --  (4, 0.1);
		\node [below] at (4, -0.1) {$2$};
		
		%\draw [dashed, red] (0, 3) -- (5, 3);		% Gestrichelte Linie
		\node [left] at (0,2) {$1$};
		{#2}\\[1ex]
		{#1}\\[1ex]
}{\end{tikzpicture}}
\fi
% Möglichst keine Ergänzungen hier, sondern in header.tex
\begin{document}
\title{\Huge{SigSys 1}}
\maketitle
\setcounter{tocdepth}{2}
\tableofcontents
\newpage

\section{Signalbeschreibung \skript{1}}
\subsection{Energie- und Leistungssignale \skript{3}}
\begin{tabular}{|l|l|l|}
	\hline
	\textbf{Klasse 1: Energiesignal} & \multicolumn{2}{|c|}{\textbf{Klasse 2: Leistungssignale}} \\
	\hline
	zeitbegrenzt oder abklingend,  	& \multicolumn{2}{|c|}{nicht zeitbregrenzt} \\ 
	einmalige Vorgänge, Impulse		& \multicolumn{2}{|c|}{}\\
	$W_n < \infty$		 	& \multicolumn{2}{|c|}{$W_n = \infty $} \\
	$P_n = 0$				& \multicolumn{2}{|c|}{$P_n \neq 0 $} \\
	\hline
							& \textbf{Klasse 2a: periodisch}	& \textbf{Klasse 2b: aperiodisch} \\
	\hline
	\parbox[c][4cm]{6cm}{\includegraphics[width=5.9cm]{./bilder/Signale/SinusAbklingend.png}} &
	\parbox[c][4cm]{6cm}{\includegraphics[width=5.9cm]{./bilder/Signale/Dreieck.png}} &
	\parbox[c][4cm]{6cm}{\includegraphics[width=5.9cm]{./bilder/Signale/Noise.png}} \\
	\hline
\end{tabular} \\

\begin{tabular}{|p{6cm}|p{6cm}|p{6cm}|}
 \hline
 	&\textbf{Zeitbereich} &  \textbf{Frequenzbereich} \\
 \hline
	 Normierte Signalenergie & $E_n = W_n =
	 \lim\limits_{T\to\infty}\,\int\limits_{-\frac{T}{2}}^\frac{T}{2}|f(t)|^2\,dt$ &
	 $E_n = \frac{1}{2\pi}\,\int\limits_{-\infty}^\infty|F(j\omega)|^2\,dt$ \\
 \hline
	 Normierte Signalleistung: & $P_n = \lim\limits_{T\to\infty}\,
	 \frac{1}{T}\int\limits_{-\frac{T}{2}}^\frac{T}{2}|f(t)|^2\,dt$ &  $P_n = \frac{1}{2\pi}\,\int\limits_{-\infty}^\infty\left(
	 \lim\limits_{T\to\infty}\frac{|F(j\omega)|^2}{T}\right)\,dt$ \\
\hline
\end{tabular}


\subsection{Mittelwerte \skript{5}}
\begin{tabular}{p{4.6cm}p{7.4cm}p{6cm}}
	Arithmetischer Mittelwert, Gleichwert, Linearer MW &
	$X_0 = \overline{X} = X_m = \frac {1} {T} \int\limits_{t_0}^{t_0+T} x(t)dt$ &
	Ist die Fläche unter der Zeitfunktion über eine Periode.
    \\
	Quadratischer MW, Leistung &
	$X^2 = \frac {1} {T} \int\limits_{t_0}^{t_0+T} x^2(t)dt$ & 
	$X^n = \frac {1} {T} \int\limits_{t_0}^{t_0+T} x^n(t)dt$ (MW $n$. Ordnung) 
	\\
	Effektivwert &
	$X = X_{\text{eff}}= \sqrt{X^2} = \sqrt{\frac{1}{T} \int\limits ^{t_0+T} _{t_0}{x^2(t)dt}}$
	& 
	\\
	Gleichrichtwert &
	$X_{|m|} = \bar{|X|} = \frac{1}{T} \int\limits_{t_0}^{t_0+T}{|x(t)| dt}$ &
    Arithm. Mittelwert der Zweiweggleichrichterschaltung
    \\
	Varianz, Standardabweichung	&
	$\text{Var}(x)=\sigma^2= \frac {1} {T} \int\limits_{-T/2}^{T/2}(x(t)-X_0)^2dt = X^2-X_0^2$ &
	Mittl. Fehler im Quadrat
	\\
\end{tabular}

\newpage

\subsection{Funktionen}
\begin{tabular}{ll}
\textbf{Autokorrelationsfunktion (AKF)}
	& ``Wie weit wird die Zukunft von der Vergangenheit geprägt?'' \\
\parbox{6cm}{
	\skript{8}\\
	\includegraphics[width=4cm]{./bilder/akf1.png}\\
	\includegraphics[width=4cm]{./bilder/akf2.png}
	} 
	& \parbox{12cm}{
	Für \textbf{Energiesignale} (Klasse 1):
	$$\varphi_{xx}(\tau) = \lim_{T\to\infty}\int\limits_{-T/2}^{T/2}
	x(t)x(t-\tau)dt=
	\lim_{T\to\infty}\int\limits_{-T/2}^{T/2} x(t+\tau)x(t)dt =
	\varphi_{xx}(-\tau)$$
	
	Für \textbf{periodische Leistungssignale} (Klasse 2a):
	$$\varphi_{xx}(\tau) = \frac {1} {T}
	\int\limits_{-T/2}^{T/2} x(t)x(t-\tau)dt 
	= \frac {1} {T} \int\limits_{-T/2}^{T/2} x(t+\tau)x(t)dt =
	\varphi_{xx}(-\tau)$$
	
	Für \textbf{nichtperiodische, stochastische Leistungssignale} (Klasse 2b):
	$$\varphi_{xx}(\tau) = \lim_{T\rightarrow\infty} \frac {1} {T}
	\int\limits_{-T/2}^{T/2} x(t)x(t-\tau)dt=\lim_{T\rightarrow\infty}\frac {1}
	{T} \int\limits_{-T/2}^{T/2} x(t+\tau)x(t)dt = \varphi_{xx}(-\tau)$$
	
	\textbf{Eigenschaften}
	\begin{itemize}
     \item $\varphi_{xx}(0) = X^2$ (Hat immer Diracstoss bei $\tau = 0$)
     \item $\varphi_{xx}(\tau)=\varphi_{xx}(\tau\pm mT)$, d.h. die
     AKF\index{Autokorrelationsfunktion} ist periodisch mit der gleichen Periode
     $T$ wie das Signal $x(t)$.
	\item $\varphi_{xx}(\tau)=\varphi_{xx}(-\tau)$: d.h. die AKF ist eine {\bf
	gerade Funktion}
	\item $\varphi_{xx}(0)\geq|\varphi_{xx}(\tau)|\quad$
	\item $\varphi_{xx}(\tau)\geq (X_0)^2-\sigma^2\quad$
   \end{itemize}
   
   $x(t) = a_k \cos(\omega t + \varphi) \Rightarrow \varphi_{xx}(t) =
   \frac{a_k^2}{2} \cos(\omega t)$\\
   $x(t) = b_k \sin(\omega t + \varphi) \Rightarrow \varphi_{xx}(t) =
   \frac{b_k^2}{2} \cos(\omega t)$\\ } \\
\hline & \\
\textbf{Kreuzkorrelationsfunktion (KKF)}
	& ``Wie ähnlich sind sich zwei Signale?'' \matlab{xcorr}\\
\parbox{6cm}{
 	\skript{11} \\
	\includegraphics[width=4cm]{./bilder/kkf.png}
	}
	& \parbox{12cm}{
	Für \textbf{Energiesignale} (Klasse 1):
	$$\varphi_{xy}(\tau) = \lim_{T\rightarrow\infty}\int\limits_{-T/2}^{T/2}
	x(t)y(t-\tau)dt =\int\limits_{-T/2}^{T/2} x(t+\tau)y(t)dt$$ 

	Für \textbf{periodische Leistungssignale} (Klasse 2a):
	$$\varphi_{xy}(\tau) = \frac {1} {T} \int\limits_{-T/2}^{T/2}  x(t)y(t-\tau)dt
	= \frac {1} {T} \int\limits_{-T/2}^{T/2}  x(t+\tau)y(t)dt$$ 
		
	Für \textbf{nichtperiodische, stochastische Leistungssignale} (Klasse 2b):
	$$\varphi_{xy}(\tau) = \lim_{T\rightarrow\infty} \frac {1} {T}
	\int\limits_{-T/2}^{T/2}x(t)y(t-\tau)dt = \lim_{T\rightarrow\infty}\frac {1}
	{T} \int\limits_{-T/2}^{T/2} x(t+\tau)y(t)dt$$  
	
	Bei Signalen mit verschiedenen Frequenzen ist $\varphi_{xy}$ immer $0$!\\
	} \\
\end{tabular}

\begin{tabular}{ll}
\textbf{Sprungfunktion \skript{16}}
	& Einschaltfunktion, Einheitssprung, Heaviside-Function \matlab{heaviside} \\
\parbox{6cm}{
	\includegraphics[width=5.5cm]{./bilder/sprungfunktion.png}
	}
	& \parbox{12cm}{
	$$u(t) = 1(t) = 
  \begin{cases}
    0 & \mbox{f"ur } t < 0,\\
    \frac{1}{2} & \mbox{f"ur } t = 0,\\
    1 & \mbox{f"ur } t > 0.\\
  \end{cases}$$	
	$$\mathcal{L:}\quad u(t) \laplace \frac1s$$
	} \\

\hline & \\
\textbf{Signumfunktion \skript{16}}
	& Vorzeichenfunktion \matlab{sign} \\
\parbox{6cm}{
	\includegraphics[width=5.5cm]{./bilder/sign.png}
	}
	
	& \parbox{12cm}{
	$$\sgn(t) =
  \begin{cases}
    -1 & \mbox{f"ur } t < 0,\\
    0 & \mbox{f"ur } t = 0,\\
    1 & \mbox{f"ur } t > 0.\\
  \end{cases}$$
	$$\mathcal{F:}\quad \sgn(t) \laplace \frac{-2j}{\omega}$$
	\\
	} \\
\hline & \\

\textbf{Rampenfunktion \skript{17}}
	& \matlab{ramp} \\
\parbox{6cm}{
	\includegraphics[width=5.5cm]{./bilder/ramp.png}
	}
	& \parbox{12cm}{
	$$r(t) = t u(t) =
  \begin{cases}
    0 & \mbox{f"ur } t \leq 0\\
    t & \mbox{f"ur } t > 0\\
  \end{cases}$$
	$$\mathcal{L:}\quad r(t) \laplace \frac{1}{s^2}$$
	\\
	} \\
\hline & \\

\textbf{Rechteckimpuls \skript{17}}
	&  \\
\parbox{6cm}{
	\includegraphics[width=5.5cm]{./bilder/rechteckimpuls.png}
	}
	& \parbox{12cm}{
	$$p_a(t) = u(t+a) -u(t-a) = 
  \begin{cases}
    1 & \mbox{f"ur } |t| < a\\
    \frac{1}{2} & \mbox{f"ur } |t| = a\\
    0 & \mbox{f"ur } |t| > a\\
  \end{cases}$$
	$$\mathcal{F:}\quad p_a(t) \laplace 2a \sinc(a \omega) = \frac{2}{\omega} \sin(a \omega)$$
	\\
	} \\
\hline & \\

\textbf{Dreieckimpuls \skript{18}}
	&  \\
\parbox{6cm}{
	\includegraphics[width=5.5cm]{./bilder/dreieckimpuls.png}
	}
	& \parbox{12cm}{
	$$\Lambda_a(t) =
  \begin{cases}
    1 - \frac{|t|}{a}& \mbox{f"ur } |t| < a\\
    0 & \mbox{f"ur } |t| \geq a\\
  \end{cases}$$
	$$\mathcal{F:}\quad \Lambda_a(t) \laplace
	a\left(\frac{\sin(\frac{a\omega}{2})}{\frac{a\omega}{2}}\right)^2 = a \sinc^2\left(\frac{a\omega}{2}\right)$$ \\
	} \\
\hline & \\

\textbf{Sincfunktion \skript{18}}
	& \matlab{sinc}  \\
\parbox{6cm}{
	\includegraphics[width=5.5cm]{./bilder/sinc.png}
	}
	& \parbox{12cm}{
	$$ \sinc_{\alpha}(t) = \frac{\sin(\alpha t)}{t} \qquad 
	\sinc(\alpha t) = \frac{\sin(\alpha t)}{\alpha t}$$
	
	$$\mathcal{F:}\quad  \sinc_{\alpha}(t) = \frac{\sin(\alpha t)}{t} \laplace
	\pi p_{\alpha}(\omega)$$
	$$\mathcal{F:}\quad  \sinc(\alpha t) = \frac{\sin(\alpha t)}{\alpha t} \laplace
	\frac{\pi}{\alpha} p_{\alpha}(\omega)$$ } \\
\end{tabular}

\begin{tabular}{ll}
\textbf{Impulsfunktion \skript{19}}
	& Diracimpuls, Diracstoss, Deltaimpuls \matlab{dirac} \\
\parbox{5cm}{
	\includegraphics[width=5cm]{./bilder/dirac.png}
	}
	& \parbox{13cm}{
	$$\delta(t) =
  \begin{cases}
    \infty & \mbox{f"ur } t = 0\\
    0 & \mbox{sonst}\\
  \end{cases}$$
	\begin{tabular}{|r|c|l|}\hline
	 1. & $\delta(at) = \frac{1}{|a|}\delta(t)$ & Skalierung\index{Skalierung}\\ \hline
	 2. & $\delta(\frac{t-t_0}{a}) = |a|\cdot\delta(t-t_0)$ & Skalierung und Verschiebung  \\ \hline
	 3. & $\delta(-t+t_0) = \delta(t-t_0)$ & symmetrisch\\ \hline
	 4. & $\delta(-t) = \delta(t)$ & $\delta(t)=\mbox{ gerade Funktion}$ \\ \hline
	 \textbf{5.} & $\int\limits_{-\infty}^{\infty}\delta(t-t_0)f(t)dt = f(t_0)$ & \textbf{Siebungseigenschaft}\\ \hline
	 6. & $\delta(t-t_0)f(t) = f(t_0)\delta(t-t_0)$ &  Abtastung\index{Abtastung}\\ \hline
	 \textbf{7.} & $\int\limits_{-\infty}^{\infty}A\cdot\delta(t)dt = A$ & \textbf{Spezialfall der Siebungseigenschaft} \\ \hline
	 8. & $\delta(t-t_0)\ast f(t) = f(t-t_0)$ & Faltung\\ \hline
	 9. & $\delta(t-t_1)\ast\delta(t-t_2) = \delta(t-t_1-t_2)$ & Faltung\index{Faltung}\\ \hline
	10. & $\delta(t)=\frac{\partial u(t)}{\partial t}$ & Ableitung des Einheitssprungs\index{Ableitung}\\ \hline
	11. & $\delta(t)=\lim\limits_{\omega\rightarrow \infty}\frac{\sin(\omega
	t)}{\pi t}$ & Definition\\ \hline 
	12. & $\mathcal{L}, \mathcal{F}:\quad \delta(t) \laplace 1$ 
		& Frequenzbereich \\ \hline
	\end{tabular}\\
	\vspace{.1cm}\\
	}
\end{tabular}


\begin{tabular}{ll}
\hline & \\
\textbf{Rauschen \skript{22}}
	& \matlab{randn} \\
\parbox{7cm}{
	\includegraphics[width=7cm]{./bilder/rauschen.png}
	}
	& \parbox{11cm}{
	Ist die Intensit"at der
	Rauschspannung "uber viele Frequenzdekaden
	gleich verteilt, so spricht man von weissem Rauschen. \\
	Signal to Noise Ratio: $\text{SNR} =
	\frac{\text{Signalleistung}}{\text{Rauschleistung}}$ (rauschfrei $ \rightarrow
	\infty$) \\ \\ Effektive Rauschspannung / -leistung
	$$U_r = \sqrt { 4 \cdot k \cdot T \cdot \Delta f \cdot R}$$
	$$P_r = k \cdot T \cdot \Delta f$$
	} \\
\hline & \\
\end{tabular}

\subsection{Amplitudenanalyse \skript{24}}
\begin{tabular}{ll}
	& ``Zeit während sich Signal in bestimmtem Amplitudenintervall aufhält'' \\
\parbox{7cm}{
	\includegraphics[width=7cm]{./bilder/amplitudenanalyse.png}
	}
	& \begin{minipage}[]{11cm}
			$$p(a) = \lim_{da\rightarrow 0}\frac{\underbrace{\sum t\left(
			a-\frac{da}{2}<x(t)\leq a+\frac{da}{2}\right)}_{dt}}{T\cdot da} = \frac{1}{T}\cdot
			\frac{1}{\partial a/ \partial t}$$
			$$\int p(a) da = 1$$
			
			\begin{tabular}{ll}
            Linearer Mittelwert 
            	& \fbox{$X_0  = \int\limits_{-\infty}^{\infty}a\cdot p(a)da$} \\ \\ 
            Mittelwert $n$. Ordnung 
            	& \fbox{$X^n = \int\limits_{-\infty}^{\infty}a^n\cdot p(a)da$} \\ \\
            \end{tabular}
      \end{minipage} \\
\end{tabular}

\newpage

\begin{center}
\textbf{Zusammenstellung verschiedener Verteilungen \skript{26}} \\
\begin{tabular}{|c|c|c|c|}\hline
Verteilung & gleichverteilt & gaussf"ormig & sinusf"ormig \\ \hline\hline
 & 
	\includegraphics[width=4.3cm]{./bilder/verteilungen-gleichvert.png}
 &
	\includegraphics[width=4.3cm]{./bilder/verteilungen-gauss.png}
 &
	\includegraphics[width=4.3cm]{./bilder/verteilungen-sinus.png}\\ \hline 
Amplitudendichte & & & \\ $p(a)=$ & $\begin{cases} \frac{1}{A}&|a-m|\leq
\frac{A}{2},\\ 0&|a-m|>\frac{A}{2}.\\ \end{cases}$ &
$\displaystyle\frac{1}{\sqrt{2\pi}\sigma}e^{\displaystyle\frac{-(a-\mu)^2}{2\sigma^2}}$ & $\begin{cases} \frac{1}{\pi\sqrt{A^2-a^2}}&|a|\leq A,\\ 0&|a|>A.\\ \end{cases}$\\ \hline  
 Wahrscheinlichkeit,& & & \\ 
dass die Amplitude $a$ & & & \\  
kleiner gleich $\alpha$ ist& & & \\ 
$P(a\!\leq\!\alpha)\!=\!\!\int\limits_{-\infty}^{\alpha}\!p(a)da=$ &
$\begin{cases}0&\alpha<m-\frac{A}{2},\\ \frac{\alpha-(m-\frac{A}{2})}{A}&
|\alpha-m|\leq\frac{A}{2} \\1&\alpha\geq m+\frac{A}{2}. \end{cases}$  &
$Q\left(\frac{\displaystyle\mu-\alpha}{\displaystyle\sigma}\right)$ &
$\begin{cases}0&\alpha\!\leq\!-A,\\
\frac{1}{\pi}\left(\frac{\pi}{2}\!+\!\sin^{-1}\!\left(\frac{a}{A}\right)\right)&
|\alpha|\!<\!A,\\1&\alpha\geq A. \end{cases}$ \\ \hline      
 & & & \\ 
Mittelwert $X_0 =$& $m$ & $\mu$ & 0 \\ 
& & & \\ \hline
& & & \\
Varianz $\operatorname{Var}(x) = $& $\dfrac{A^2}{12}$
& $\sigma^2$ & $\dfrac{A^2}{2}$ \\ & & & \\ \hline
& & & \\
Leistung $X^2 =$& $m^2+\dfrac{A^2}{12}$ &
$\mu^2+\sigma^2$ & $\dfrac{A^2}{2}$ \\ 
& & & \\ \hline
\end{tabular}
\end{center}
\textbf{Anmerkung zur gaussförmigen Verteilung:} Im Intervall $\mu \pm 3\sigma$
sind 99,73\% aller Messwerte zu finden. In der Zeichung ist diese Stelle mit
\textbf{b} gekennzeichnet.

\newpage

\begin{tabular}{ll}
\textbf{Faltung \skript{28}}
	& Convolution, ``Addition zweier unabhängiger ergodischer Prozesse $n_i$'' \matlab{conv} \\
\parbox{5cm}{
	\includegraphics[width=5cm]{./bilder/faltung.png}
	\\}
	& \parbox{13cm}{
	$p(a) =
	\int\limits_{-\infty}^{\infty}p_1(\xi)\cdot p_2(a-\xi) d\xi = p_1(a)
	\ast p_2(a) =  p_2(a) \ast p_1(a) = \int\limits_{-\infty}^{\infty}p_2(\xi)\cdot 
  	p_1(a-\xi) d\xi$ \\
  	Die Breite des Faltungsproduktes entspricht der Summe der Breite der
  	einzelnen Faktoren.\\ \\
  	Faltung im Zeitbereich $\rightarrow$ Multiplikation im Frequenzbereich 
  	$$f(t) * g(t) \laplace F(s) G(s)$$
  	Faltung im Frequenzbereich $\rightarrow$ Multiplikation im Zeitbereich
  	$$F(s) * G(s) \Laplace \frac{1}{2 \pi} f(t) g(t)$$ } \\
\end{tabular}

\begin{tabular}{ll}
\hline & \\
\textbf{Q-Funktion \skript{35}}
	& ``Wahrscheinlichkeit eines Fehlers'' \matlab{erf, erfc} \\
\parbox{6cm}{
	\includegraphics[width=5cm]{./bilder/q-funktion.png}
	}
	& \parbox{12cm}{
		Wenn die Resultate einer Messserie mit einer Normalverteilung mit Varianz
		$\sigma$ und Erwartungswert $0$ auftreten, dann ist
		$\operatorname{erf}\,\left(\,\frac{a}{\sigma \sqrt{2}}\,\right)$ die
		Wahrscheinlichkeit, dass ein einzelner Messwert zwischen $-a$ und $a$ liegt. 
		\\
		Tabelle \skript{44}\\
		$$Q(\xi)=\frac{1}{\sqrt{2\pi}}\int\limits_{\xi}^{\infty}
		e^{-\frac{y^2}{2}}dy$$
		$$Q(\xi) = \frac12 \operatorname{erfc}\left(\frac{\xi}{\sqrt2}\right)
		= \frac12 \left(1 - \operatorname{erf}\left( \frac{\xi}{\sqrt2}\right) \right)
		$$ }
\end{tabular}

\newpage
\section{Signalflussdiagramm \skript{55}}
	\begin{list}{$\bullet$}{\setlength{\itemsep}{0cm} \setlength{\parsep}{0cm} \setlength{\topsep}{0cm}} 
	  \item Graphische Lösung linearer Gleichungen
	  \item Graphische Darstellung von LTI-Systemen
	  \item Änderung der Topologie ohne UTF zu ändern
	\end{list}
	
	\subsection{Glossar \skript{56}}
	  \begin{tabular}{|m{5cm} | m{13cm}|}
	    \hline
		   \textbf{Knoten:} &
		   Darstellung einer Grösse, eines Signals oder einer Variable 
		   \\
		   Quelle: &
		   Unabhängiger Knoten, es münden keine Zweige ein 
		   \\
		   Senke: &
		   Knoten, ohne weggehende Zweige 
		   \\
   		\hline
	      \textbf{Zweig:} &
	      Funktionelle Abhängigkeit einer Grösse 
	      \\
	    \hline
	      \textbf{Pfad:} &
	      Kontinuierliche Folge von Zweigen, die in die gleiche Richtung zeigen \\
	      Offener Pfad: &
	      Ein Pfad, bei dem jeder beteiligte Knoten nur \textbf{einmal} durchquert wird 
	      \\
	      Vorwärtspfad: &
	      Ein offener Pfad zwischen einer Quelle und einer Senke 
	      \\
   		\hline
	      \textbf{Schleife (L):} &
	      Ein geschlossener Pfad, welcher zum Ausgangsknoten zurückkehrt, 
	      wobei jeder beteiligte Knoten nur einmal durchlaufen wird, ausgenommen der
	      Ausgangsknoten \\
	      Eigenschleife: &
	      Eine (Rückkopplungs)schleife, die aus einem Zweig und einem Knoten besteht \\
	    \hline
	      \textbf{Zweigtransmittanz:} &
	      Die lineare Grösse, unabhängig von ihrer Dimension, 
	      die einen Knoten eines Zweiges zum anderen Knoten in Beziehung setzt. \\
	    \hline
	      \textbf{Schleifentransmittanz:} &
	      Das Produkt der Zweigtransmittanzen in einer Schleife. \\
	    \hline
	  \end{tabular}
  
	\subsection{Konstruktionsregel \skript{61}}
		\begin{itemize}
			\item Knoten = Variablen und Zweigtransmittanzen = Koeffizienten des linearen Gleichungssystem.
			\item Signale durchqueren Zweige nur in Pfeilrichtung und werden mit der entsprechenden Zweigtransmittanz multipliziert.
			\item Wert der Variable (Knoten) = Summe aller Signale, die in diesen Knoten einmünden.
			\item Wert der Variable (Knoten) wird auf alle weggehenden Zweige übertragen.
		\end{itemize}
	
		\textbf{Beispiel:}
		
		\begin{multicols}{2}
			$X_1(z) = a \cdot X_4(z) + X_0(z)$	\\
			$X_2(z) = X_1(z)$	\\
			$X_3(z) = b_0 \cdot X_2(z) + b_1 \cdot X_4(z)$	\\
			$X_4(z) = z^(-1) \cdot X_2(z)$	\\
			$X_5(z) = X_3(z)$
			\vfill
			\columnbreak
			\vfill
			\includegraphics[width=8cm]{./bilder/schleifenkonstruktion_beispiel.png}
		\end{multicols}
	  
	\subsection{Reduktionsregel \skript{61}}
	    \begin{multicols}{2}
	      \subsubsection{Regel 1: Kettentransformation \skript{61}}
		      Die gesamte Übertragung einer Kaskade von Zweigen (d.h. einem Pfad) ist gleich dem Produkt der einzelnen Transmittanzen. \\
		      \includegraphics[width=8cm]{./bilder/kettentransformation.png}
	      
	      \subsubsection{Regel 2: Paralleltransformation \skript{61}}
		      Die gesamte Transmittanz paralleler Zweige ist gleich der Summe der einzelnen Zweigtransmittanzen. \\
		      \includegraphics[width=8cm]{./bilder/paralleltransformation.png}
	      
	      \subsubsection{Regel 3: Entfernung eines Knotens \skript{61}}
		      Der Anfangs- oder Endpunkt einer Transmittanz kann entfernt oder verschoben werden, solange die Transmittanz zwischen den interessierenden Knoten im System unverändert bleibt. \\
		      \includegraphics[width=8cm]{./bilder/entfernungeinesknotens.png}

	      \subsubsection{Regel 4: Transmittanzverschiebung \skript{62}}
	        \includegraphics[width=7cm]{./bilder/transmittanzverschiebung.png} \\
	        Wichtig ist, dass eine neue Variable $x_3'$ eingeführt wird, wenn der Endpunkt eines
	        inneren Zweiges verschoben wird. \newline (siehe c)    
	    
	      \subsubsection{Regel 5: Pfadinversion \skript{63}}
	        \includegraphics[width=8cm]{./bilder/pfadinversion.png} \\
	        Es gilt zu beachten, das die Inversion eines Pfaden (dessen Anfangspunkt nach Definition
	        eine Quelle sein muss) den Effekt hat, dass die Quelle vom einen Ende des Pfades zum anderen
	        Ende verschoben wird. Der Pfad von $x_i$ nach $x_j$ hat eine Transmittanz von $L$. 
	        Den zu invertierenden Pfad setzten wir $\frac{1}{L}$ und alle Pfade welche ursprünglich in
	        $x_i$ endeten, werden verschoben, dass sie neu in $x_j$ enden und ihre Transmittanzen werden
	        mit $-\frac{1}{L}$ multipliziert.
	        
	      \subsubsection{Regel 6: Entfernen einer Eigenschleife \skript{64}}
	        \includegraphics[width=8cm]{./bilder/eigenschleife.png} \\
	        Die Eigenschleife hat die Transmittanz $L$. Sie wird entfernt indem man bei allen anderen
	        Zweigen welche \textbf{in den Knoten münden}, durch $(1-L)$ \textbf{dividiert}.
	        
	      \subsubsection{Regel 7: Schleifenreduktion \skript{65}}
		      \paragraph{Einzelne Schleife} \ \\
			      Die Transmittanz einer unabhängigen Variablen $x_i$ (d. h. einer Quelle) zu einer abhängigen Variable (d. h. einem inneren Knoten oder einer Senke) in einem SFD, das nur eine Schleife und einen Vorwärtspfad enthält, ist gleich $H_{ij}=\frac{P_{ij}}{1-L}$ wobei $P_{ij}$ die Transmittanz des Vorwärtspfades von $x_i$ nach $x_j$ und $L$ die Transmittanz der Schleife ist. Die Formel lässt sich mittels der Lösung des entsprechenden Gleichungssystems oder äquivalent durch Transmittanzverschiebung und Entfernung der Eigenschleife beweisen. \\
			      \includegraphics[width=10cm]{./bilder/schleifenreduktion_einzelneschleife.png}
			      
			  \paragraph{Mehrere sich nicht berührende Schleifen} \ \\
				  Bei einer Kaskade von sich nicht berührenden Schleifen (d. h. , dass sie keine jeweils gemeinsamen Knoten haben) ist die gesamte Transmittanz gleich dem Produkt der einzelnen Transmittanzen: $H_{in}=\frac{P_{ij}}{1-L_j}\cdot\frac{P_{jk}}{1-L_k}\cdot\ldots\cdot\frac{P_{(n-1)n}}{1-L_n}$ \\
				  \includegraphics[width=10cm]{./bilder/schleifenreduktion_nichtberuehrend.png} \\
				  $H_{15}=\frac{x_5}{x_1}=\frac{P_{13}}{1-L_1}\cdot\frac{P_{35}}{1-L_2}=\frac{P_{15}}{1-L_1-L_2+L_1\cdot L_2}=\frac{abcd}{1-be-df+bedf}$
				  
			  \paragraph{Mehrere sich berührende Schleifen} \ \\
			      Für den Fall, dass die zwei Schleifen mindestens einen gemeinsamen Knoten haben, ist die gesamte Transmittanz gegeben durch: $H_{ij}=\frac{P_{ij}}{1-L_i-L_j}$ \\
			      \includegraphics[width=10cm]{./bilder/schleifenreduktion_beruehrend.png} \\
			      $H_{15}=\frac{x_5}{x_1}=\frac{P_{15}}{1-L_1-L_2}=\frac{abcd}{1-be-cf}$
			      
	      \subsubsection{Regel 8: Allgemeine Mehrfachschleifen-Reduktionsregel für einfache Pfade \skript{66}}
		       Wir betrachten hier den Mehrfachschleifenfall wo nur ein Pfad zwischen einer Quelle $x_i$ und einem abhängigen Knoten $x_j$ existiert, wobei dieser Pfad jede Schleife im SFD berührt, (d. h. dass er wenigstens einen Knoten mit jeder Schleife gemeinsam hat). Kurz ausgedrückt, lautet die Mehrfachschleifen Reduktionsregel für einfache Pfade: $H_{ij}=\frac{P_{ij}}{\Delta}$ \\
		       Die Grösse $\delta$ ist die Graph- oder Netzwerkdeterminante. $\Delta$ wird folgendermassen ermittelt:\\
		       $\Delta=1-(\text{Summe alle Schleifen})+(\text{Summe aller Produkte zweier Schleifen, die sich nicht berühren})-(\text{Summe aller Produkte dreier Schleifen, die sich nicht berühren})+\ldots$ \\
		       \includegraphics[width=10cm]{./bilder/mehrfachreduktion_1.png} \\
		       $H_{17}=\frac{x_7}{x_1}=\frac{P_{17}}{\Delta}=\frac{P_{17}}{1-(L_1+L_2+L_3+L_4+L_5)+(L_1L_3+L_1L_4+L_1L_5+L_2L_4+L_2L_5+L_3L_5)-L_1L_3L_5}=\frac{abcdef}{1-(bg+ch+di+ej+fk)+(bgdi+bgej+bgfk+chej+chfk+difk)-bgdifk}$ \\
		       \includegraphics[width=10cm]{./bilder/mehrfachreduktion_2.png} \\
		       $H_{06}=\frac{x_6}{x_0}=\frac{P_{06}}{\Delta}=\frac{P_{06}}{1-(L_1+L_2+L_3)+(L_1L_2+L_1L_3+L_2L_3)-L_1L_2L_3}=\frac{abcde}{1-af-cg-eh+afcg+afeh+cgeh-afcgeh}$
	   \end{multicols}
	   \newpage
	   
\subsection{Mason's Regel \skript{69}}
	\begin{multicols}{2}
		$\boxed{H_{ij} = \frac{\sum\limits_k P_k\cdot\Delta_k}{\Delta}}\quad$ 
		\vfill
		\columnbreak
		\vfill
		\boldmath{$P_k$}: Vorwärtspfad $k$ (bez. auf 1 Eingang) \enspace \\
		\boldmath{$\Delta_k$}: Kofaktor des $k$-ten Pfades \enspace \\
		\boldmath{$\Delta$}: Netzwerkdet/Graphdet\vspace{0.3cm}
		\vfill
		\columnbreak
		\vfill
	\end{multicols}
	\textbf{Hinweis:} UTF von $x_i$ nach $x_j$, wobei \textbf{$x_i$} eine
	\textbf{Quelle}, \textbf{$x_j$} jedoch nicht zwingend eine \textbf{Senke} sein
	muss. \vspace{0.3cm}\\
	\boldmath$\Delta_k$ = 1 - (Summe aller Schleifen die $P_k$ nicht berühren) + (Summe aller Produkte zweier Schleifen, die $P_k$ und sich selbst nicht
	berühren) - (Summe aller Produkte dreier Schleifen, die $P_k$ und sich selbst
	nicht berühren) + \unboldmath $\ldots$\vspace{0.3cm} \\
	\boldmath$\Delta$ = 1 - (Summe aller Schleifen) + (Summe aller Produkte zweier
	Schleifen, die sich nicht berühren) - (Summe
	aller Produkte dreier Schleifen, die sich nicht berühren) +\unboldmath$\ldots$ \\
	
	Falls die \textbf{UTF eines SFD von einem beliebigen Knoten} (keiner Quelle)
	gesucht wird, kann Mason's Regel nicht direkt angewandt werden. Abhilfe: \\
	$\boxed{H_{ij} = \frac{x_j}{x_i} = \frac{x_j}{x_q} \frac{x_q}{x_i} =
	\frac{H_{qj}}{H_{qi}}}$ Wobei $x_q$ eine Quelle sei. 
	Schlussendlich kürzt sich die Netzwerkdeterminante heraus. \\

\begin{minipage}[t]{10cm}
\subsection{Beispiel eines SFD \skript{73}}
	\includegraphics[width=7cm]{./bilder/sfd-bsp.png} 
    \begin{enumerate}
		\item[a)] Die UTF\index{UTF} zwischen $X_1$ und $X_4$ ist (mit Mason's Regel):\index{Mason's Regel}\\
		$H_{14}=\frac{X_4}{X_1}=\frac{aeh+abc(1-g)}{1-ef-g}$
		\item[b)] Das folgende Gleichungssystem beschreibt das SFD.\\
		$X_2 =a\cdot X_1+f\cdot X_3$\\
		$X_3 =e\cdot X_2+g\cdot X_3$\\
		$X_4 =h\cdot X_3+c\cdot X_6$\\
		$X_5 =d\cdot X_4$\\
		$X_6 =b\cdot X_2$
		\item Nach Umformung der Gleichungen erhalten wir: \\
		$X_4=h\cdot X_3+\frac{bc}{e}\cdot (1-g)\cdot X_3\quad\&\quad X_3\cdot
		\frac{1-g}{e}=a\cdot X_1+f\cdot X3.$ 
		\item Somit ist \\ $X_4=\frac{h+\frac{bc}{e}(1-g)}{\frac{1-g}{ae}-\frac{f}{a}}X_1=\frac{aeh+abc(1-g)}{1-g-ef}X_1$.
	\end{enumerate}
	
\subsection{Fundamentales SFD \skript{74}}
	\includegraphics[width=7cm]{./bilder/sfd-ordnung.png} \\
		\textit{Ordnung eines SFD = Anzahl der fundamentalen Knoten}: Knoten, welche
		entfernt werden müssen, um \textit{alle} Schleifen aufzubrechen.
\end{minipage}
\begin{minipage}[t]{8cm}
\subsubsection{Fundamentales SFD erster Ordnung \skript{75}}
	\includegraphics[width=3cm]{./bilder/sfd-fundamental-erster-ordnung.png} \\
	Durch Reduzieren auf das fundamentale SFD 1. Ordnung, kann die UTF direkt
	ermittelt werden: \\
	\[ H_{io} = \frac{E_o}{E_i}=
	t_{io}+\frac{t_{ix}t_{xo}}{1-t_{xx}}=
	\frac{t_{io}-t_{io}t_{xx}+t_{ix}t_{xo}}{1-t_{xx}} \] \\
	\begin{tabular}{l l p{7cm}}
		$E_x$ & = & Fundamentaler Knoten\\
		$E_i$ & = & Quelle (Eingang) \\
		$E_o$ & = & Senke (Ausgang) \\
		$t_{xx}$ & = & alle Eigenschleifen des Knoten $E_x$ \\
		$t_{ix}$ & = & alle Pfade von der Quelle zum Knoten $E_x$ \\
		$t_{xo}$ & = &alle Pfade vom Knoten $E_x$ zur Senke \\
		$t_{io}$ & = & Leckpfad, alle Pfade von der Quelle zur Senke, welche \underline{nicht}
		durch den Knoten $E_x$ führen.
	\end{tabular}\\
	Wenn es mehrere Wege gibt, dann zusammen zählen: Bsp.: $tix = tix_1 +
	tix_2$
\end{minipage}

\subsection{Einbezug analoger Verstärker \skript{80}}
\includegraphics[width=18cm]{./bilder/sfd-op.png}

\subsection{Inversion \skript{84}}
	Durch schrittweise Pfadinversion erhält man das ”invertierte” Signalflussdiagramm. \\
	Es hat folgende Eigenschaften:
	\begin{multicols}{2}
		\begin{itemize}
			\item Richtungsumdrehung aller Zweigtransmittanzen bei gleichbleibenden Transmittanzen
			\item Spiegelung des resultierenden SFD
			\item Bezeichnungswechsel von Eingangs- und Ausgangsknoten
		\end{itemize}
		\columnbreak
		\includegraphics[width=9cm]{./bilder/inversion.png}
	\end{multicols}

\subsection{Transposition \skript{84}}
 	\begin{multicols}{2}
		\textbf{Ablauf:}
		\begin{enumerate}
		  \item Richtungsumdrehung aller Zweigtransmittanzen bei gleichbleibenden Transmittanzen
		  \item Spiegelung des resultierenden SFD
		  \item Bezeichnungswechsel von Eingangs- und Ausgangsknoten
		\end{enumerate}
    
    Die UTF des transponierten SFD ist \textbf{identisch} mit der UTF des ursprünglichen SFD, aber ihre
    Topologie ist verschieden.
    
  \columnbreak
    \includegraphics[width=9cm]{./bilder/transposition.png}
  \end{multicols}

\subsection{Skalierung \skript{85}}
	\vspace*{-1cm}
	\begin{minipage}[]{10cm}
		Um einen oder mehrere Knoten zu ändern, ohne das gesamte System zu
		ändern (Voraussetzung: Start/Endknoten werden nicht mitmaskiert), kann man
		diese Knoten skalieren.\\
		Vorgehen: 
		\begin{enumerate}
                \item Skalierungszone festlegen (Trennbündel) $N_b$
                \item Alle eingehende Zweige mit $\lambda$ multiplizieren
                \item Alle ausgehende Zweige mit $\frac{1}{\lambda}$
                multiplizieren
	    \end{enumerate}
	    Wenn alle maximalen Signalniveaus gleich $\rightarrow$ maximal möglichen
	    Dynamikbereich\\
	\end{minipage}
	\hspace*{1cm}
    \begin{minipage}[t]{7cm}
      \includegraphics[width=7cm]{./bilder/sfd-scalierung.png}
       Die Skalierung kann verwendet werden um den Dynamikbereich zu maximieren, Inverter zu entfernen
       und die Verstärkung und Signalniveaus innerhalb eines Systems zu ändern.
    \end{minipage}

    


\section{Frequenzanalyse \skript{132}}

\subsection{Wichtige Theoreme}
	\begin{tabular}{|l|l|}
		\hline
			Bessel's Theorem \skript{124} &
			$\int\limits_{-\infty}^{\infty}|f(t)|^2 dt = \frac{1}{2\pi} \int\limits_{-\infty}^{\infty}|F(j\omega)|^2d\omega$\\
		\hline
			Parseval's Theorem \skript{124} &
			$\int\limits_{-\infty}^{\infty}f(t)\cdot g^*(t)dt = \frac{1}{2\pi}\int\limits_{-\infty}^{\infty}F(j\omega)
			\cdot G^*(j\omega) d\omega$\\
		\hline
			Gibbschesph"anomen \skript{112} &
			"Uberschwinger betr"agt ca. $18\%$ der Amplitude oder ca. $9\%$ der Sprungh"ohe.\\
			& $S_{\infty} = \frac{f(x_0^+)+f(x_0^-)}{2} \qquad$ (approximiert)\\
		\hline
			Autokorrelation \skript{132} &
			$\varphi_{xx}(\tau) = \sum\limits_{k=-\infty}^{\infty}c_kc_{-k}e^{-j\frac{2\pi k}{T_0}\tau} =
			\sum\limits_{k=-\infty}^{\infty}|c_k|^2 e^{-j\frac{2\pi k}{T_0}\tau} =
			|c_0|^2 + 2\cdot \sum\limits_{k=1}^{\infty} |c_k|^2 \cdot \cos(\frac{2\pi k}{T_0}\tau)$ \\
		\hline
			Leistung \skript{119} &
			$X^2 = \sum\limits_{k=-\infty}^{\infty} |c_k|^2 = |c_0|^2 + 2\cdot \sum\limits_{k=1}^{\infty} |c_k|^2 =
			(\frac{a_0}{2})^2 + \sum\limits_{k=1}^{\infty} \frac{a_k^2 + b_k^2}{2} =
			(\frac{a_0}{2})^2 + \sum\limits_{k=1}^{\infty}\frac{A_k^2}{2}$\\
		\hline
			Bandbreitentheorem \skript{122} &
			$\Delta\omega \cdot \Delta t \geq \gamma \qquad \text{mit } \gamma \geq \frac{1}{2}$\\
		\hline
	\end{tabular}
	
\subsection{Leisungsdichtespektrum \skript{132}}
	\begin{tabular}{p{6cm} l}
		$\phi(j\omega) = \lim\limits_{T\to\infty} \frac{|F(j\omega)|^2}{T}$ &
		$\phi(j\omega)$: Leistungsdichtespektrum\\
		
		$P_n = \frac{1}{2\pi} \int\limits_{-\infty}^{\infty}\phi(j\omega)d\omega$ &
		$P_n$: normierte Leistung\\
		
		$E(j\omega) = |F(j\omega)|^2$ &
		$E(j\omega)$: Energiedichtespektrum
	\end{tabular}
	
\begin{minipage}[]{10cm}
\subsection{Wiener-Chintchine Theorem \skript{133}}
	Leistungssignal 2a: \\
	$\varphi_{xx}(t) = \frac{1}{2\pi}\int\limits_{-\infty}^{\infty}\phi(j\omega)e^{j\omega t}d\omega \; \laplace \; \phi(j\omega) = \int\limits_{-\infty}^{\infty}\varphi_{xx}(t)e^{-j\omega t} dt \nonumber $\\
	Energiesignal: \\
	$\varphi_{xx}(t) = \frac{1}{2\pi}\int\limits_{-\infty}^{\infty}E(j\omega)e^{j\omega t}d\omega \; \laplace \; E(j\omega) = \int\limits_{-\infty}^{\infty}\varphi_{xx}(t)e^{-j\omega t} dt \nonumber $\\
	Leistungssignal 2b: \\
	$\varphi_{xy}(t) = \frac{1}{2\pi}\int\limits_{-\infty}^{\infty}\phi_{xy}(j\omega)e^{j\omega t}d\omega \; \laplace \;
	\phi_{xy}(j\omega) = \int\limits_{-\infty}^{\infty}\varphi_{xy}(t)e^{-j\omega t} dt \nonumber$
\end{minipage}
\begin{minipage}[]{8cm}
\subsection{Eigenschaften von $\phi(j\omega)$}
	\begin{enumerate}
		\item	$\phi(j\omega)$ ist reell
		\item $\phi(j\omega) \geq 0$
		\item $\phi(j\omega) = \phi(-j\omega)$
		\item $P = X^2 = \varphi_{xx}(0) = \frac{1}{2\pi} \int\limits_{-\infty}^{\infty} \phi(j\omega) d\omega$
		\item $\phi(0) = \int\limits_{-\infty}^{\infty} \varphi_{xx}(\tau)d\tau$
	\end{enumerate}
\end{minipage}
\section{Systeme \skript{95}}
\subsection{Begriffe}
\begin{tabular}{|p{5cm}|p{5cm}|p{7.5cm}|}
\hline & & \\
\textit{Bezeichnung}
	& \textit{Beschreibung}
	& \textit{Bedingung, Erkennung} \\
\hline \hline & & \\
Wirkungsfreiheit \skript{95}
	& Eingang des System hochohmig 
	& Kaskadierte Systeme durch Einheitsverstärker verbunden\\
\hline & & \\
Statische bzw. dynamische Systeme \skript{96}
	& Ohne bzw. mit Gedächtnis
	& Statisch: $u_2(t)$ nur vom Eingangssignal $u_1(t)$ bei $t$ abhängig \\
	& & Dynamisch: $\int dt; \; \frac{d}{dt}; \; f(t \pm t_0) $\\
\hline & & \\
Kausale bzw. akausale Systeme \skript{99}
	& Keine zukünftigen Werte bzw. nicht in ``Echtzeit''
	& \parbox{7cm}{Kausal: $f(t - t_0); \int^t f(\tau) d \tau \quad (t_0 > 0) \quad$ \\ Statische
	Systeme sind immer kausal. \\ \\ Akausal: $f(-t); \; f(t + t_0); \; \int^{t+t_0}
	f(\tau) d \tau$} \\
	\hline & & \\ Zeitinvariante bzw. zeitvariante Systeme  \skript{104} & Von der Zeit (un-) abhängig
	& \parbox{7cm}{Zeitvariant: $\cos(t) x(t); t^{\alpha} x(t) \quad \text{(} \alpha \neq 0 \text{)} $
	\\ \\ Zeitinvariant: $S(x(t-t_0)=S(x)\cdot x(t-t_0)$} \\
\hline & & \\
Lineare bzw. nichtlineare Systeme \skript{105}
	&
	& \parbox{7cm}{Nichtlinear: $f^{\alpha}(t); \; \alpha + f(t) $ \\ Kennlinie nicht durch
	Ursprung.\\ \\ Linear: $S(x1+x2)=S(x1)+S(x2)$ \\ $S(c\cdot x)=c\cdot S(x)$} \\
\hline 
\end{tabular}

\subsection{Übertragungsfunktion von LTI-Systemen \skript{108}}
\begin{tabular}{ll}
\parbox{13cm}{
	$$h(t) \laplace H(s)$$
	$$s_2(t) = h(t) * s_1(t) \laplace S_2(s) = H(s) S_1(s)$$}
	& \parbox{5cm}{
		\includegraphics[width=5cm]{./bilder/utf-theorie.png}}\\
\multicolumn{2}{l}{Kaskadierung von wirkungsfreien Systemen:
$H_{total}(s) = H_1(s) H_2(s)$ bzw. bei $n$ gleichen Systemen:
$H_{total} = (H(s))^n$}
 \\ \\

\parbox{13cm}{
	\textbf{Beispiel}: Gesucht UTF $H(s) = \frac{Y(s)}{X(s)}$ \\
	$$H(s) = \frac{sL}{\frac{1}{sC} + sL + R} = \frac{s^2}{\frac{1}{LC} + s
	\frac{R}{L} + s^2}$$\\
	$$\Longrightarrow \text{Pole bei } s = -\frac{R}{2L} \pm j
	\sqrt{\frac{1}{LC} - \left(\frac{R}{2L}\right)^2} \quad ; \quad \text{Doppelte
	Nullstelle bei } s = 0$$
	Differentialgleichung:    $ \ddot{y}(t)+
	\frac{R}{L}\dot{y}(t)+\frac{1}{LC}y=\ddot{x}(t)$}
	
	
	& \parbox{5cm}{
		\includegraphics[width=5cm]{./bilder/utf-beispiel.png}} \\
		
\end{tabular}
\begin{multicols}{2}
\subsubsection{Bestimmung der UTF}
\begin{tabular}{ll}
	Bauteil & Ersatz \\
	R & R\\
	L & sL\\
	C & $\frac{1}{sC}$ \\
\end{tabular}\\
Parallelschaltung $= \frac{1}{\frac{1}{R}+\frac{1}{sL}+sC}$\\
Das Potential in einem Punkt berechnet man mit $=\frac{Summe\ aller\
Elemente\ zwischen\ Punkt\ und\ GND}{Summe\ aller\ Elemente\ der\ Kompletten\
Schaltung}$\\
Die Ordnung der UTF ist die Anzahl unabhängiger Speicher (L oder C).\\
\end{multicols}

\subsection{Berechnung des Amplituden- und Phasengangs aus der
Übertragungsfunktion}
$$H(j \omega) = \frac{Y(j \omega)}{X(j \omega)} = \underbrace{|H(j
\omega)|}_{Amplitudengang} e^{j\underbrace{\Theta(\omega)}_{Phasengang}} = 
\frac{|Y(j \omega)|}{|X(j \omega)|} e^{j (\arg(Y(j \omega)) - \arg(X(j
\omega)))} =
\frac{|Y(j \omega)|}{|X(j \omega)|} e^{j \left[\arctan \left(\frac{\Imag\{Y(j
\omega)\}}{\Real\{Y(j \omega)\}} \right) - \arctan \left(\frac{\Imag\{X(j
\omega)\}}{\Real\{X(j \omega)\}} \right)\right]}$$

\subsection{Zusammenhang zwischen Impuls- \& Einheitssprungantwort, Endwerte
\skript{109}}
$ \text{Einheitssprungantwort } g(t) \text{, Impulsantwort }h(t)$
$$h(t)= \frac{\partial g(t)}{\partial t}\quad\text{bzw.}\quad
g(t)=\int_{-\infty}^{t}h(\tau)d\tau \qquad;\qquad 
\lim\limits_{t \rightarrow \infty}  h(t)= \lim\limits_{s \rightarrow 0} s H(s)
\qquad;\qquad
\lim\limits_{t \rightarrow \infty}  g(t)= \lim\limits_{s \rightarrow 0} H(s)$$

\subsection{Phasen- \& Gruppenlaufzeit \skript{116}}
\definecolor{gruppe}{rgb}{1,.75,0} % 255,192,0
\definecolor{phase}{rgb}{1,0,0} % 255,0,0
Die \textcolor{phase}{Phasenlaufzeit}  ist nur für reine Sinussignale bestimmbar:
$\tau_P(\omega)=\frac{-\theta(\omega)}{\omega}$ \\
Die \textcolor{gruppe}{Gruppenlaufzeit} hingegen ist für sämtliche Signale möglich:
$\tau_G(\omega)=\frac{-\partial\theta(\omega)}{\partial\omega}$\\
\includegraphics[width=18cm]{./bilder/laufzeit.png}\\
Eingangssignal $x(t)$ und Ausgangssignal $y(t)$ des Systems
$H(s)=\frac{1}{s^2+0.2s+1}$. Bemerkung: $y(t)$ ist gr"osser als $x(t)$.

\subsection{Stabilität von LTI-Systemen}
\subsubsection{Asymptotische Stabilität \skript{112}}
\begin{tabular}{ll}
Stabil: 
	& $\lim\limits_{t\rightarrow\infty} h(t) = 0$ 
	\qquad Pole \textbf{nur} in der
linken s-Halbebene.\\
Instabil: 
	& Mind. ein Pol in der rechten s-Halbebene oder mind. ein
\textbf{mehrfacher} Pol auf der $j$-Achse der s-Ebene. \\
Grenzstabil:
	& mindestens ein \textbf{einfacher Pol} auf der $j$-Achse
\end{tabular}

\subsubsection{Stabilität mit Hurwitz-Polynom \skript{113}}
Es wird jeweils das Polynom im \textbf{Nenner der Übertragungsfunktion} betrachtet:
$P(s) = a_n s^n + a_{n-1} s^{n-1} +\ldots +a_1s + a_0$ \\
\begin{tabular}{|l||l| l|}\hline
$N$   &   $P(s)$ ist ein Hurwitz-Polynom (stabil) &  $P(s)$ ist ein
modifiziertes Hurwitz-Polynom (grenzstabil) \\ \hline\hline
      1     &      gilt f"ur alle $P(s)$          &  $a_0=0$ \\ \hline
      2     &     gilt f"ur alle $P(s)$           &  $a_1=0$ \\ \hline
      3     &     $a_1a_2>a_0a_3$      &  $a_1a_2=a_0a_3$ \\ \hline
      4     &     $a_3(a_1a_2-a_0a_3)>a_1^2a_4$   &    $a_3(a_1a_2-a_0a_3)=a_1^2a_4$\\ \hline

      5    &     {\footnotesize $a_3a_4>a_2a_5$  und}   &     {\footnotesize $a_3a_4>a_2a_5$} \\
           &     {\footnotesize
           $(a_1a_2-a_0a_3)(a_3a_4-a_2a_5)>(a_1a_4-a_0a_5)^2$}   &  
           {\footnotesize $(a_1a_2-a_0a_3)(a_3a_4-a_2a_5)=(a_1a_4-a_0a_5)^2$} 
           \\ \hline   
\end{tabular}\\
\begin{itemize}
  \item Wenn \textbf{ein Koeffizient negativ} ist $(a_x < 0)$, dann ist das System
  \textbf{instabil}.
  \item Wenn \textbf{alle Koeffizienten negativ} sind, kann $-1$ ausgeklammert
  werden und in den Zähler verschoben werden\\ $\Rightarrow$ \textbf{System
  stabil} oder \textbf{grenzstabil} (siehe Punkt 3)
  \item Wenn \textbf{ein Koeffizient nicht vorhanden} ist $(a_x = 0)$, dann ist das System
  evtl. grenzstabil, d.h. es ist eine \textbf{Überprüfung mit modifiziertem Hurwitz-Polynom}
  nötig.
\end{itemize}

\subsection{Klirrfaktor \skript{122}}
Als Mass für nichtlineare Verzerrungen gilt der \textit{Klirrfaktor}. Betrachtet
wird jeweils der Effektivwert am Ausgang 
$$k = \sqrt{\frac{U_2^2 + U_3^2 + \ldots + U_n^2}{U_1^2 + U_2^2 + \ldots +
U_n^2}} \qquad 0 \leq k \leq 1$$ 
\begin{tabular}{ll}
Teilklirrfaktor (frequenzselektiv) 
	&$k_m =  \frac {U_m} {\sqrt{ U_1^2+ U_2^2 + \ldots + U_n^2} }$ \\
Klirrdämpfungsmass 
	& $a_k = 20 \log \left( \frac1k \right)$ $\qquad$ \\
Teilklirrdämpfungmass 
	& $a_k = 20 \log \left( \frac{1}{k_m} \right)$
\end{tabular}

\subsection{Total Harmonic Distortion (THD) \skript{122}}
$$\text{THD} = \sqrt{ \frac {U_2^2+ U_3^2 + \ldots + U_n^2} {U_1^2} } \qquad
\infty > \text{THD} \geq k \geq 0; \quad \text{Für kleine Verzerrungen: THD}
\approx k $$
\section{Einheitssignale}
	
	\subsection{Funktionen}
	\renewcommand{\arraystretchOriginal}{1}
	\begin{tabular}{ll}
		\textbf{Sprungfunktion \skript{16}} &
		Einschaltfunktion, Einheitssprung, Heaviside-Function \matlab{heaviside}
		\\
		\parbox{6cm}{
			\includegraphics[width=5.5cm]{./bilder/sprungfunktion.png}
		}
		&
		\parbox{12cm}{
			$$u(t) = 1(t) = 
  			\begin{cases}
    			0 & \mbox{f"ur } t < 0,\\
    			\frac{1}{2} & \mbox{f"ur } t = 0,\\
    			1 & \mbox{f"ur } t > 0.\\
  			\end{cases}$$	
			$$\mathcal{L:}\quad u(t) \; \laplace \; \frac1s$$
		}
		\\ \hline & \\
		
		\textbf{Signumfunktion \skript{16}} &
		Vorzeichenfunktion \matlab{sign}
		\\
		\parbox{6cm}{
			\includegraphics[width=5.5cm]{./bilder/sign.png}
		}
		&
		\parbox{12cm}{
			$$\sgn(t) =
  			\begin{cases}
    			-1 & \mbox{f"ur } t < 0,\\
    			0 & \mbox{f"ur } t = 0,\\
    			1 & \mbox{f"ur } t > 0.\\
  			\end{cases}$$
			$$\mathcal{F:}\quad \sgn(t) \; \laplace \; \frac{-2j}{\omega}$$ \\
		}
		\\ \hline & \\

		\textbf{Rampenfunktion \skript{17}} & \matlab{ramp}
		\\
		\parbox{6cm}{
			\includegraphics[width=5.5cm]{./bilder/ramp.png}
		}
		&
		\parbox{12cm}{
			$$r(t) = t u(t) =
  			\begin{cases}
    			0 & \mbox{f"ur } t \leq 0\\
    			t & \mbox{f"ur } t > 0\\
  			\end{cases}$$
			$$\mathcal{L:}\quad r(t) \; \laplace \; \frac{1}{s^2}$$ \\
		}
		\\ \hline & \\

		\textbf{Rechteckimpuls \skript{17}} & 
		\\
		\parbox{6cm}{
			\includegraphics[width=5.5cm]{./bilder/rechteckimpuls.png}
		}
		&
		\parbox{12cm}{
			$$p_a(t) = u(t+a) -u(t-a) = 
  			\begin{cases}
    			1 & \mbox{f"ur } |t| < a\\
    			\frac{1}{2} & \mbox{f"ur } |t| = a\\
    			0 & \mbox{f"ur } |t| > a\\
  			\end{cases}$$
			$$\mathcal{F:}\quad p_a(t) \; \laplace \; 2a \sinc(a \omega) = \frac{2}{\omega} \sin(a \omega)$$ \\
		}
		\\ \hline & \\

		\textbf{Dreieckimpuls \skript{18}} & 
		\\
		\parbox{6cm}{
			\includegraphics[width=5.5cm]{./bilder/dreieckimpuls.png}
		}
		&
		\parbox{12cm}{
			$$\Lambda_a(t) =
  			\begin{cases}
    			1 - \frac{|t|}{a}& \mbox{f"ur } |t| < a\\
    			0 & \mbox{f"ur } |t| \geq a\\
  			\end{cases}$$
			$$\mathcal{F:}\quad \Lambda_a(t) \; \laplace \;
			a\left(\frac{\sin(\frac{a\omega}{2})}{\frac{a\omega}{2}}\right)^2 = a \sinc^2\left(\frac{a\omega}{2}\right)$$ \\
		}
		\\ \hline & \\

		\textbf{Sincfunktion \skript{18}} &
		\matlab{sinc}
		\\
		\parbox{6cm}{
			\includegraphics[width=5.5cm]{./bilder/sinc.png}
		}
		&
		\parbox{12cm}{
			$$ \sinc_{\alpha}(t) = \frac{\sin(\alpha t)}{t} \qquad 
			\sinc(\alpha t) = \frac{\sin(\alpha t)}{\alpha t}$$
			$$\mathcal{F:}\quad  \sinc_{\alpha}(t) = \frac{\sin(\alpha t)}{t} \; \laplace \;
			\pi p_{\alpha}(\omega)$$
			$$\mathcal{F:}\quad  \sinc(\alpha t) = \frac{\sin(\alpha t)}{\alpha t} \; \laplace \;
			\frac{\pi}{\alpha} p_{\alpha}(\omega)$$ } \\
		\end{tabular}

		\begin{tabular}{ll}
		\textbf{Impulsfunktion \skript{19}} &
		Diracimpuls, Diracstoss, Deltaimpuls \matlab{dirac}
		\\
		\parbox{5cm}{
			\includegraphics[width=5cm]{./bilder/dirac.png}
		}
		&
		\parbox{13cm}{
			$$\delta(t) =
  			\begin{cases}
    			\infty & \mbox{f"ur } t = 0\\
    			0 & \mbox{sonst}\\
  			\end{cases}$$
		\begin{tabular}{|r|c|l|} \hline
	 		1. & $\delta(at) = \frac{1}{|a|}\delta(t)$ & Skalierung\index{Skalierung}\\ \hline
	 		2. & $\delta(\frac{t-t_0}{a}) = |a|\cdot\delta(t-t_0)$ & Skalierung und Verschiebung  \\ \hline
	 		3. & $\delta(-t+t_0) = \delta(t-t_0)$ & symmetrisch\\ \hline
	 		4. & $\delta(-t) = \delta(t)$ & $\delta(t)=\mbox{ gerade Funktion}$ \\ \hline
	 		\textbf{5.} & $\int\limits_{-\infty}^{\infty}\delta(t-t_0)f(t)dt = f(t_0)$ & \textbf{Siebungseigenschaft}\\ \hline
	 		6. & $\delta(t-t_0)f(t) = f(t_0)\delta(t-t_0)$ &  Abtastung\index{Abtastung}\\ \hline
	 		\textbf{7.} & $\int\limits_{-\infty}^{\infty}A\cdot\delta(t)dt = A$ & \textbf{Spezialfall der Siebungseigenschaft} \\ \hline
	 		8. & $\delta(t-t_0)\ast f(t) = f(t-t_0)$ & Faltung\\ \hline
	 		9. & $\delta(t-t_1)\ast\delta(t-t_2) = \delta(t-t_1-t_2)$ & Faltung\index{Faltung}\\ \hline
			10. & $\delta(t)=\frac{\partial u(t)}{\partial t}$ & Ableitung des Einheitssprungs\index{Ableitung}\\ \hline
			11. & $\delta(t)=\lim\limits_{\omega\rightarrow \infty}\frac{\sin(\omega t)}{\pi t}$ & Definition\\ \hline 
			12. & $\mathcal{L}, \mathcal{F}:\quad \delta(t) \; \laplace \; 1$ 
			& Frequenzbereich
		\\ \hline
		\end{tabular}
		\\
		\vspace{.1cm}\\
		}
		\end{tabular}


%	\subsection{Eigenschaften unterschiedlicher Schwingungsformen}
%
%		\begin{center}
%		\rotatebox{90}{
%		\begin{minipage}[]{24cm}
%			\includegraphics[width=24cm]{./bilder/schwingungsformen.png}\\
%			\textbf{Hinweis zur Rechtecksfunktion:} Für ganzzahlige Tastverhältnisse $\frac{T}{t_{on}} = n$
%			verschwindet die \textbf{n. Harmonische} Schwingung und all deren Vielfache!
%		\end{minipage}}
%		\end{center}	

	\input{idiotenseite/elektrotechnik/subsections/signalformen}	
\section{Tabelle der wichtigsten Ableitungen und Integrale}
\input{table/diverses/subsections/Ableitungen}
\input{table/diverses/subsections/unbestimmteIntegrale}
\input{table/elektrotechnik/subsections/signalformen}
\includepdf[pages={1-9}]{Tabellen.pdf}

\end{document}
