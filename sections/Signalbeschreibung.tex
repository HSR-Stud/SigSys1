\section{Signalbeschreibung \skript{1}}
\subsection{Energie- und Leistungssignale \skript{3}}
\begin{tabular}{|l|l|l|}
	\hline
	\textbf{Klasse 1: Energiesignal} & \multicolumn{2}{|c|}{\textbf{Klasse 2: Leistungssignale}} \\
	\hline
	zeitbegrenzt oder abklingend,  	& \multicolumn{2}{|c|}{nicht zeitbregrenzt} \\ 
	einmalige Vorgänge, Impulse		& \multicolumn{2}{|c|}{}\\
	$W_n < \infty$		 	& \multicolumn{2}{|c|}{$W_n = \infty $} \\
	$P_n = 0$				& \multicolumn{2}{|c|}{$P_n \neq 0 $} \\
	\hline
							& \textbf{Klasse 2a: periodisch}	& \textbf{Klasse 2b: aperiodisch} \\
	\hline
	\parbox[c][4cm]{6cm}{\includegraphics[width=5.9cm]{./bilder/Signale/SinusAbklingend.png}} &
	\parbox[c][4cm]{6cm}{\includegraphics[width=5.9cm]{./bilder/Signale/Dreieck.png}} &
	\parbox[c][4cm]{6cm}{\includegraphics[width=5.9cm]{./bilder/Signale/Noise.png}} \\
	\hline
\end{tabular} \\

\begin{tabular}{|p{6cm}|p{6cm}|p{6cm}|}
 \hline
 	&\textbf{Zeitbereich} &  \textbf{Frequenzbereich} \\
 \hline
	 Normierte Signalenergie & $E_n = W_n =
	 \lim\limits_{T\to\infty}\,\int\limits_{-\frac{T}{2}}^\frac{T}{2}|f(t)|^2\,dt$ &
	 $E_n = \frac{1}{2\pi}\,\int\limits_{-\infty}^\infty|F(j\omega)|^2\,dt$ \\
 \hline
	 Normierte Signalleistung: & $P_n = \lim\limits_{T\to\infty}\,
	 \frac{1}{T}\int\limits_{-\frac{T}{2}}^\frac{T}{2}|f(t)|^2\,dt$ &  $P_n = \frac{1}{2\pi}\,\int\limits_{-\infty}^\infty\left(
	 \lim\limits_{T\to\infty}\frac{|F(j\omega)|^2}{T}\right)\,dt$ \\
\hline
\end{tabular}


\subsection{Mittelwerte \skript{5}}
\begin{tabular}{p{4.6cm}p{7.4cm}p{6cm}}
	Arithmetischer Mittelwert, Gleichwert, Linearer MW &
	$X_0 = \overline{X} = X_m = \frac {1} {T} \int\limits_{t_0}^{t_0+T} x(t)dt$ &
	Ist die Fläche unter der Zeitfunktion über eine Periode.
    \\
	Quadratischer MW, Leistung &
	$X^2 = \frac {1} {T} \int\limits_{t_0}^{t_0+T} x^2(t)dt$ & 
	$X^n = \frac {1} {T} \int\limits_{t_0}^{t_0+T} x^n(t)dt$ (MW $n$. Ordnung) 
	\\
	Effektivwert &
	$X = X_{\text{eff}}= \sqrt{X^2} = \sqrt{\frac{1}{T} \int\limits ^{t_0+T} _{t_0}{x^2(t)dt}}$
	& 
	\\
	Gleichrichtwert &
	$X_{|m|} = \bar{|X|} = \frac{1}{T} \int\limits_{t_0}^{t_0+T}{|x(t)| dt}$ &
    Arithm. Mittelwert der Zweiweggleichrichterschaltung
    \\
	Varianz, Standardabweichung	&
	$\text{Var}(x)=\sigma^2= \frac {1} {T} \int\limits_{-T/2}^{T/2}(x(t)-X_0)^2dt = X^2-X_0^2$ &
	Mittl. Fehler im Quadrat
	\\
\end{tabular}

\newpage

\subsection{Funktionen}
\begin{tabular}{ll}
\textbf{Autokorrelationsfunktion (AKF)}
	& ``Wie weit wird die Zukunft von der Vergangenheit geprägt?'' \\
\parbox{6cm}{
	\skript{8}\\
	\includegraphics[width=4cm]{./bilder/akf1.png}\\
	\includegraphics[width=4cm]{./bilder/akf2.png}
	} 
	& \parbox{12cm}{
	Für \textbf{Energiesignale} (Klasse 1):
	$$\varphi_{xx}(\tau) = \lim_{T\to\infty}\int\limits_{-T/2}^{T/2}
	x(t)x(t-\tau)dt=
	\lim_{T\to\infty}\int\limits_{-T/2}^{T/2} x(t+\tau)x(t)dt =
	\varphi_{xx}(-\tau)$$
	
	Für \textbf{periodische Leistungssignale} (Klasse 2a):
	$$\varphi_{xx}(\tau) = \frac {1} {T}
	\int\limits_{-T/2}^{T/2} x(t)x(t-\tau)dt 
	= \frac {1} {T} \int\limits_{-T/2}^{T/2} x(t+\tau)x(t)dt =
	\varphi_{xx}(-\tau)$$
	
	Für \textbf{nichtperiodische, stochastische Leistungssignale} (Klasse 2b):
	$$\varphi_{xx}(\tau) = \lim_{T\rightarrow\infty} \frac {1} {T}
	\int\limits_{-T/2}^{T/2} x(t)x(t-\tau)dt=\lim_{T\rightarrow\infty}\frac {1}
	{T} \int\limits_{-T/2}^{T/2} x(t+\tau)x(t)dt = \varphi_{xx}(-\tau)$$
	
	\textbf{Eigenschaften}
	\begin{itemize}
     \item $\varphi_{xx}(0) = X^2$ (Hat immer Diracstoss bei $\tau = 0$)
     \item $\varphi_{xx}(\tau)=\varphi_{xx}(\tau\pm mT)$, d.h. die
     AKF\index{Autokorrelationsfunktion} ist periodisch mit der gleichen Periode
     $T$ wie das Signal $x(t)$.
	\item $\varphi_{xx}(\tau)=\varphi_{xx}(-\tau)$: d.h. die AKF ist eine {\bf
	gerade Funktion}
	\item $\varphi_{xx}(0)\geq|\varphi_{xx}(\tau)|\quad$
	\item $\varphi_{xx}(\tau)\geq (X_0)^2-\sigma^2\quad$
   \end{itemize}
   
   $x(t) = a_k \cos(\omega t + \varphi) \Rightarrow \varphi_{xx}(t) =
   \frac{a_k^2}{2} \cos(\omega t)$\\
   $x(t) = b_k \sin(\omega t + \varphi) \Rightarrow \varphi_{xx}(t) =
   \frac{b_k^2}{2} \cos(\omega t)$\\ } \\
\hline & \\
\textbf{Kreuzkorrelationsfunktion (KKF)}
	& ``Wie ähnlich sind sich zwei Signale?'' \matlab{xcorr}\\
\parbox{6cm}{
 	\skript{11} \\
	\includegraphics[width=4cm]{./bilder/kkf.png}
	}
	& \parbox{12cm}{
	Für \textbf{Energiesignale} (Klasse 1):
	$$\varphi_{xy}(\tau) = \lim_{T\rightarrow\infty}\int\limits_{-T/2}^{T/2}
	x(t)y(t-\tau)dt =\int\limits_{-T/2}^{T/2} x(t+\tau)y(t)dt$$ 

	Für \textbf{periodische Leistungssignale} (Klasse 2a):
	$$\varphi_{xy}(\tau) = \frac {1} {T} \int\limits_{-T/2}^{T/2}  x(t)y(t-\tau)dt
	= \frac {1} {T} \int\limits_{-T/2}^{T/2}  x(t+\tau)y(t)dt$$ 
		
	Für \textbf{nichtperiodische, stochastische Leistungssignale} (Klasse 2b):
	$$\varphi_{xy}(\tau) = \lim_{T\rightarrow\infty} \frac {1} {T}
	\int\limits_{-T/2}^{T/2}x(t)y(t-\tau)dt = \lim_{T\rightarrow\infty}\frac {1}
	{T} \int\limits_{-T/2}^{T/2} x(t+\tau)y(t)dt$$  
	
	Bei Signalen mit verschiedenen Frequenzen ist $\varphi_{xy}$ immer $0$!\\
	} \\
\end{tabular}

\begin{tabular}{ll}
\textbf{Sprungfunktion \skript{16}}
	& Einschaltfunktion, Einheitssprung, Heaviside-Function \matlab{heaviside} \\
\parbox{6cm}{
	\includegraphics[width=5.5cm]{./bilder/sprungfunktion.png}
	}
	& \parbox{12cm}{
	$$u(t) = 1(t) = 
  \begin{cases}
    0 & \mbox{f"ur } t < 0,\\
    \frac{1}{2} & \mbox{f"ur } t = 0,\\
    1 & \mbox{f"ur } t > 0.\\
  \end{cases}$$	
	$$\mathcal{L:}\quad u(t) \laplace \frac1s$$
	} \\

\hline & \\
\textbf{Signumfunktion \skript{16}}
	& Vorzeichenfunktion \matlab{sign} \\
\parbox{6cm}{
	\includegraphics[width=5.5cm]{./bilder/sign.png}
	}
	
	& \parbox{12cm}{
	$$\sgn(t) =
  \begin{cases}
    -1 & \mbox{f"ur } t < 0,\\
    0 & \mbox{f"ur } t = 0,\\
    1 & \mbox{f"ur } t > 0.\\
  \end{cases}$$
	$$\mathcal{F:}\quad \sgn(t) \laplace \frac{-2j}{\omega}$$
	\\
	} \\
\hline & \\

\textbf{Rampenfunktion \skript{17}}
	& \matlab{ramp} \\
\parbox{6cm}{
	\includegraphics[width=5.5cm]{./bilder/ramp.png}
	}
	& \parbox{12cm}{
	$$r(t) = t u(t) =
  \begin{cases}
    0 & \mbox{f"ur } t \leq 0\\
    t & \mbox{f"ur } t > 0\\
  \end{cases}$$
	$$\mathcal{L:}\quad r(t) \laplace \frac{1}{s^2}$$
	\\
	} \\
\hline & \\

\textbf{Rechteckimpuls \skript{17}}
	&  \\
\parbox{6cm}{
	\includegraphics[width=5.5cm]{./bilder/rechteckimpuls.png}
	}
	& \parbox{12cm}{
	$$p_a(t) = u(t+a) -u(t-a) = 
  \begin{cases}
    1 & \mbox{f"ur } |t| < a\\
    \frac{1}{2} & \mbox{f"ur } |t| = a\\
    0 & \mbox{f"ur } |t| > a\\
  \end{cases}$$
	$$\mathcal{F:}\quad p_a(t) \laplace 2a \sinc(a \omega) = \frac{2}{\omega} \sin(a \omega)$$
	\\
	} \\
\hline & \\

\textbf{Dreieckimpuls \skript{18}}
	&  \\
\parbox{6cm}{
	\includegraphics[width=5.5cm]{./bilder/dreieckimpuls.png}
	}
	& \parbox{12cm}{
	$$\Lambda_a(t) =
  \begin{cases}
    1 - \frac{|t|}{a}& \mbox{f"ur } |t| < a\\
    0 & \mbox{f"ur } |t| \geq a\\
  \end{cases}$$
	$$\mathcal{F:}\quad \Lambda_a(t) \laplace
	a\left(\frac{\sin(\frac{a\omega}{2})}{\frac{a\omega}{2}}\right)^2 = a \sinc^2\left(\frac{a\omega}{2}\right)$$ \\
	} \\
\hline & \\

\textbf{Sincfunktion \skript{18}}
	& \matlab{sinc}  \\
\parbox{6cm}{
	\includegraphics[width=5.5cm]{./bilder/sinc.png}
	}
	& \parbox{12cm}{
	$$ \sinc_{\alpha}(t) = \frac{\sin(\alpha t)}{t} \qquad 
	\sinc(\alpha t) = \frac{\sin(\alpha t)}{\alpha t}$$
	
	$$\mathcal{F:}\quad  \sinc_{\alpha}(t) = \frac{\sin(\alpha t)}{t} \laplace
	\pi p_{\alpha}(\omega)$$
	$$\mathcal{F:}\quad  \sinc(\alpha t) = \frac{\sin(\alpha t)}{\alpha t} \laplace
	\frac{\pi}{\alpha} p_{\alpha}(\omega)$$ } \\
\end{tabular}

\begin{tabular}{ll}
\textbf{Impulsfunktion \skript{19}}
	& Diracimpuls, Diracstoss, Deltaimpuls \matlab{dirac} \\
\parbox{5cm}{
	\includegraphics[width=5cm]{./bilder/dirac.png}
	}
	& \parbox{13cm}{
	$$\delta(t) =
  \begin{cases}
    \infty & \mbox{f"ur } t = 0\\
    0 & \mbox{sonst}\\
  \end{cases}$$
	\begin{tabular}{|r|c|l|}\hline
	 1. & $\delta(at) = \frac{1}{|a|}\delta(t)$ & Skalierung\index{Skalierung}\\ \hline
	 2. & $\delta(\frac{t-t_0}{a}) = |a|\cdot\delta(t-t_0)$ & Skalierung und Verschiebung  \\ \hline
	 3. & $\delta(-t+t_0) = \delta(t-t_0)$ & symmetrisch\\ \hline
	 4. & $\delta(-t) = \delta(t)$ & $\delta(t)=\mbox{ gerade Funktion}$ \\ \hline
	 \textbf{5.} & $\int\limits_{-\infty}^{\infty}\delta(t-t_0)f(t)dt = f(t_0)$ & \textbf{Siebungseigenschaft}\\ \hline
	 6. & $\delta(t-t_0)f(t) = f(t_0)\delta(t-t_0)$ &  Abtastung\index{Abtastung}\\ \hline
	 \textbf{7.} & $\int\limits_{-\infty}^{\infty}A\cdot\delta(t)dt = A$ & \textbf{Spezialfall der Siebungseigenschaft} \\ \hline
	 8. & $\delta(t-t_0)\ast f(t) = f(t-t_0)$ & Faltung\\ \hline
	 9. & $\delta(t-t_1)\ast\delta(t-t_2) = \delta(t-t_1-t_2)$ & Faltung\index{Faltung}\\ \hline
	10. & $\delta(t)=\frac{\partial u(t)}{\partial t}$ & Ableitung des Einheitssprungs\index{Ableitung}\\ \hline
	11. & $\delta(t)=\lim\limits_{\omega\rightarrow \infty}\frac{\sin(\omega
	t)}{\pi t}$ & Definition\\ \hline 
	12. & $\mathcal{L}, \mathcal{F}:\quad \delta(t) \laplace 1$ 
		& Frequenzbereich \\ \hline
	\end{tabular}\\
	\vspace{.1cm}\\
	}
\end{tabular}


\begin{tabular}{ll}
\hline & \\
\textbf{Rauschen \skript{22}}
	& \matlab{randn} \\
\parbox{7cm}{
	\includegraphics[width=7cm]{./bilder/rauschen.png}
	}
	& \parbox{11cm}{
	Ist die Intensit"at der
	Rauschspannung "uber viele Frequenzdekaden
	gleich verteilt, so spricht man von weissem Rauschen. \\
	Signal to Noise Ratio: $\text{SNR} =
	\frac{\text{Signalleistung}}{\text{Rauschleistung}}$ (rauschfrei $ \rightarrow
	\infty$) \\ \\ Effektive Rauschspannung / -leistung
	$$U_r = \sqrt { 4 \cdot k \cdot T \cdot \Delta f \cdot R}$$
	$$P_r = k \cdot T \cdot \Delta f$$
	} \\
\hline & \\
\end{tabular}

\subsection{Amplitudenanalyse \skript{24}}
\begin{tabular}{ll}
	& ``Zeit während sich Signal in bestimmtem Amplitudenintervall aufhält'' \\
\parbox{7cm}{
	\includegraphics[width=7cm]{./bilder/amplitudenanalyse.png}
	}
	& \begin{minipage}[]{11cm}
			$$p(a) = \lim_{da\rightarrow 0}\frac{\underbrace{\sum t\left(
			a-\frac{da}{2}<x(t)\leq a+\frac{da}{2}\right)}_{dt}}{T\cdot da} = \frac{1}{T}\cdot
			\frac{1}{\partial a/ \partial t}$$
			$$\int p(a) da = 1$$
			
			\begin{tabular}{ll}
            Linearer Mittelwert 
            	& \fbox{$X_0  = \int\limits_{-\infty}^{\infty}a\cdot p(a)da$} \\ \\ 
            Mittelwert $n$. Ordnung 
            	& \fbox{$X^n = \int\limits_{-\infty}^{\infty}a^n\cdot p(a)da$} \\ \\
            \end{tabular}
      \end{minipage} \\
\end{tabular}

\newpage

\begin{center}
\textbf{Zusammenstellung verschiedener Verteilungen \skript{26}} \\
\begin{tabular}{|c|c|c|c|}\hline
Verteilung & gleichverteilt & gaussf"ormig & sinusf"ormig \\ \hline\hline
 & 
	\includegraphics[width=4.3cm]{./bilder/verteilungen-gleichvert.png}
 &
	\includegraphics[width=4.3cm]{./bilder/verteilungen-gauss.png}
 &
	\includegraphics[width=4.3cm]{./bilder/verteilungen-sinus.png}\\ \hline 
Amplitudendichte & & & \\ $p(a)=$ & $\begin{cases} \frac{1}{A}&|a-m|\leq
\frac{A}{2},\\ 0&|a-m|>\frac{A}{2}.\\ \end{cases}$ &
$\displaystyle\frac{1}{\sqrt{2\pi}\sigma}e^{\displaystyle\frac{-(a-\mu)^2}{2\sigma^2}}$ & $\begin{cases} \frac{1}{\pi\sqrt{A^2-a^2}}&|a|\leq A,\\ 0&|a|>A.\\ \end{cases}$\\ \hline  
 Wahrscheinlichkeit,& & & \\ 
dass die Amplitude $a$ & & & \\  
kleiner gleich $\alpha$ ist& & & \\ 
$P(a\!\leq\!\alpha)\!=\!\!\int\limits_{-\infty}^{\alpha}\!p(a)da=$ &
$\begin{cases}0&\alpha<m-\frac{A}{2},\\ \frac{\alpha-(m-\frac{A}{2})}{A}&
|\alpha-m|\leq\frac{A}{2} \\1&\alpha\geq m+\frac{A}{2}. \end{cases}$  &
$Q\left(\frac{\displaystyle\mu-\alpha}{\displaystyle\sigma}\right)$ &
$\begin{cases}0&\alpha\!\leq\!-A,\\
\frac{1}{\pi}\left(\frac{\pi}{2}\!+\!\sin^{-1}\!\left(\frac{a}{A}\right)\right)&
|\alpha|\!<\!A,\\1&\alpha\geq A. \end{cases}$ \\ \hline      
 & & & \\ 
Mittelwert $X_0 =$& $m$ & $\mu$ & 0 \\ 
& & & \\ \hline
& & & \\
Varianz $\operatorname{Var}(x) = $& $\dfrac{A^2}{12}$
& $\sigma^2$ & $\dfrac{A^2}{2}$ \\ & & & \\ \hline
& & & \\
Leistung $X^2 =$& $m^2+\dfrac{A^2}{12}$ &
$\mu^2+\sigma^2$ & $\dfrac{A^2}{2}$ \\ 
& & & \\ \hline
\end{tabular}
\end{center}
\textbf{Anmerkung zur gaussförmigen Verteilung:} Im Intervall $\mu \pm 3\sigma$
sind 99,73\% aller Messwerte zu finden. In der Zeichung ist diese Stelle mit
\textbf{b} gekennzeichnet.

\newpage

\begin{tabular}{ll}
\textbf{Faltung \skript{28}}
	& Convolution, ``Addition zweier unabhängiger ergodischer Prozesse $n_i$'' \matlab{conv} \\
\parbox{5cm}{
	\includegraphics[width=5cm]{./bilder/faltung.png}
	\\}
	& \parbox{13cm}{
	$p(a) =
	\int\limits_{-\infty}^{\infty}p_1(\xi)\cdot p_2(a-\xi) d\xi = p_1(a)
	\ast p_2(a) =  p_2(a) \ast p_1(a) = \int\limits_{-\infty}^{\infty}p_2(\xi)\cdot 
  	p_1(a-\xi) d\xi$ \\
  	Die Breite des Faltungsproduktes entspricht der Summe der Breite der
  	einzelnen Faktoren.\\ \\
  	Faltung im Zeitbereich $\rightarrow$ Multiplikation im Frequenzbereich 
  	$$f(t) * g(t) \laplace F(s) G(s)$$
  	Faltung im Frequenzbereich $\rightarrow$ Multiplikation im Zeitbereich
  	$$F(s) * G(s) \Laplace \frac{1}{2 \pi} f(t) g(t)$$ } \\
\end{tabular}

\begin{tabular}{ll}
\hline & \\
\textbf{Q-Funktion \skript{35}}
	& ``Wahrscheinlichkeit eines Fehlers'' \matlab{erf, erfc} \\
\parbox{6cm}{
	\includegraphics[width=5cm]{./bilder/q-funktion.png}
	}
	& \parbox{12cm}{
		Wenn die Resultate einer Messserie mit einer Normalverteilung mit Varianz
		$\sigma$ und Erwartungswert $0$ auftreten, dann ist
		$\operatorname{erf}\,\left(\,\frac{a}{\sigma \sqrt{2}}\,\right)$ die
		Wahrscheinlichkeit, dass ein einzelner Messwert zwischen $-a$ und $a$ liegt. 
		\\
		Tabelle \skript{44}\\
		$$Q(\xi)=\frac{1}{\sqrt{2\pi}}\int\limits_{\xi}^{\infty}
		e^{-\frac{y^2}{2}}dy$$
		$$Q(\xi) = \frac12 \operatorname{erfc}\left(\frac{\xi}{\sqrt2}\right)
		= \frac12 \left(1 - \operatorname{erf}\left( \frac{\xi}{\sqrt2}\right) \right)
		$$ }
\end{tabular}

\newpage