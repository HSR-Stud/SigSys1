\section{Frequenzanalyse \skript{132}}

\subsection{Wichtige Theoreme}
	\begin{tabular}{|l|l|}
		\hline
			Bessel's Theorem \skript{124} &
			$\int\limits_{-\infty}^{\infty}|f(t)|^2 dt = \frac{1}{2\pi} \int\limits_{-\infty}^{\infty}|F(j\omega)|^2d\omega$\\
		\hline
			Parseval's Theorem \skript{124} &
			$\int\limits_{-\infty}^{\infty}f(t)\cdot g^*(t)dt = \frac{1}{2\pi}\int\limits_{-\infty}^{\infty}F(j\omega)
			\cdot G^*(j\omega) d\omega$\\
		\hline
			Gibbschesph"anomen \skript{112} &
			"Uberschwinger betr"agt ca. $18\%$ der Amplitude oder ca. $9\%$ der Sprungh"ohe.\\
			& $S_{\infty} = \frac{f(x_0^+)+f(x_0^-)}{2} \qquad$ (approximiert)\\
		\hline
			Autokorrelation \skript{132} &
			$\varphi_{xx}(\tau) = \sum\limits_{k=-\infty}^{\infty}c_kc_{-k}e^{-j\frac{2\pi k}{T_0}\tau} =
			\sum\limits_{k=-\infty}^{\infty}|c_k|^2 e^{-j\frac{2\pi k}{T_0}\tau} =
			|c_0|^2 + 2\cdot \sum\limits_{k=1}^{\infty} |c_k|^2 \cdot \cos(\frac{2\pi k}{T_0}\tau)$ \\
		\hline
			Leistung \skript{119} &
			$X^2 = \sum\limits_{k=-\infty}^{\infty} |c_k|^2 = |c_0|^2 + 2\cdot \sum\limits_{k=1}^{\infty} |c_k|^2 =
			(\frac{a_0}{2})^2 + \sum\limits_{k=1}^{\infty} \frac{a_k^2 + b_k^2}{2} =
			(\frac{a_0}{2})^2 + \sum\limits_{k=1}^{\infty}\frac{A_k^2}{2}$\\
		\hline
			Bandbreitentheorem \skript{122} &
			$\Delta\omega \cdot \Delta t \geq \gamma \qquad \text{mit } \gamma \geq \frac{1}{2}$\\
		\hline
	\end{tabular}
	
\subsection{Leisungsdichtespektrum \skript{132}}
	\begin{tabular}{p{6cm} l}
		$\phi(j\omega) = \lim\limits_{T\to\infty} \frac{|F(j\omega)|^2}{T}$ &
		$\phi(j\omega)$: Leistungsdichtespektrum\\
		
		$P_n = \frac{1}{2\pi} \int\limits_{-\infty}^{\infty}\phi(j\omega)d\omega$ &
		$P_n$: normierte Leistung\\
		
		$E(j\omega) = |F(j\omega)|^2$ &
		$E(j\omega)$: Energiedichtespektrum
	\end{tabular}
	
\begin{minipage}[]{10cm}
\subsection{Wiener-Chintchine Theorem \skript{133}}
	Leistungssignal 2a: \\
	$\varphi_{xx}(t) = \frac{1}{2\pi}\int\limits_{-\infty}^{\infty}\phi(j\omega)e^{j\omega t}d\omega \; \laplace \; \phi(j\omega) = \int\limits_{-\infty}^{\infty}\varphi_{xx}(t)e^{-j\omega t} dt \nonumber $\\
	Energiesignal: \\
	$\varphi_{xx}(t) = \frac{1}{2\pi}\int\limits_{-\infty}^{\infty}E(j\omega)e^{j\omega t}d\omega \; \laplace \; E(j\omega) = \int\limits_{-\infty}^{\infty}\varphi_{xx}(t)e^{-j\omega t} dt \nonumber $\\
	Leistungssignal 2b: \\
	$\varphi_{xy}(t) = \frac{1}{2\pi}\int\limits_{-\infty}^{\infty}\phi_{xy}(j\omega)e^{j\omega t}d\omega \; \laplace \;
	\phi_{xy}(j\omega) = \int\limits_{-\infty}^{\infty}\varphi_{xy}(t)e^{-j\omega t} dt \nonumber$
\end{minipage}
\begin{minipage}[]{8cm}
\subsection{Eigenschaften von $\phi(j\omega)$}
	\begin{enumerate}
		\item	$\phi(j\omega)$ ist reell
		\item $\phi(j\omega) \geq 0$
		\item $\phi(j\omega) = \phi(-j\omega)$
		\item $P = X^2 = \varphi_{xx}(0) = \frac{1}{2\pi} \int\limits_{-\infty}^{\infty} \phi(j\omega) d\omega$
		\item $\phi(0) = \int\limits_{-\infty}^{\infty} \varphi_{xx}(\tau)d\tau$
	\end{enumerate}
\end{minipage}