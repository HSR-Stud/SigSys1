\section{Einheitssignale}
	
	\subsection{Funktionen}
	\renewcommand{\arraystretchOriginal}{1}
	\begin{tabular}{ll}
		\textbf{Sprungfunktion \skript{16}} &
		Einschaltfunktion, Einheitssprung, Heaviside-Function \matlab{heaviside}
		\\
		\parbox{6cm}{
			\includegraphics[width=5.5cm]{./bilder/sprungfunktion.png}
		}
		&
		\parbox{12cm}{
			$$u(t) = 1(t) = 
  			\begin{cases}
    			0 & \mbox{f"ur } t < 0,\\
    			\frac{1}{2} & \mbox{f"ur } t = 0,\\
    			1 & \mbox{f"ur } t > 0.\\
  			\end{cases}$$	
			$$\mathcal{L:}\quad u(t) \; \laplace \; \frac1s$$
		}
		\\ \hline & \\
		
		\textbf{Signumfunktion \skript{16}} &
		Vorzeichenfunktion \matlab{sign}
		\\
		\parbox{6cm}{
			\includegraphics[width=5.5cm]{./bilder/sign.png}
		}
		&
		\parbox{12cm}{
			$$\sgn(t) =
  			\begin{cases}
    			-1 & \mbox{f"ur } t < 0,\\
    			0 & \mbox{f"ur } t = 0,\\
    			1 & \mbox{f"ur } t > 0.\\
  			\end{cases}$$
			$$\mathcal{F:}\quad \sgn(t) \; \laplace \; \frac{-2j}{\omega}$$ \\
		}
		\\ \hline & \\

		\textbf{Rampenfunktion \skript{17}} & \matlab{ramp}
		\\
		\parbox{6cm}{
			\includegraphics[width=5.5cm]{./bilder/ramp.png}
		}
		&
		\parbox{12cm}{
			$$r(t) = t u(t) =
  			\begin{cases}
    			0 & \mbox{f"ur } t \leq 0\\
    			t & \mbox{f"ur } t > 0\\
  			\end{cases}$$
			$$\mathcal{L:}\quad r(t) \; \laplace \; \frac{1}{s^2}$$ \\
		}
		\\ \hline & \\

		\textbf{Rechteckimpuls \skript{17}} & 
		\\
		\parbox{6cm}{
			\includegraphics[width=5.5cm]{./bilder/rechteckimpuls.png}
		}
		&
		\parbox{12cm}{
			$$p_a(t) = u(t+a) -u(t-a) = 
  			\begin{cases}
    			1 & \mbox{f"ur } |t| < a\\
    			\frac{1}{2} & \mbox{f"ur } |t| = a\\
    			0 & \mbox{f"ur } |t| > a\\
  			\end{cases}$$
			$$\mathcal{F:}\quad p_a(t) \; \laplace \; 2a \sinc(a \omega) = \frac{2}{\omega} \sin(a \omega)$$ \\
		}
		\\ \hline & \\

		\textbf{Dreieckimpuls \skript{18}} & 
		\\
		\parbox{6cm}{
			\includegraphics[width=5.5cm]{./bilder/dreieckimpuls.png}
		}
		&
		\parbox{12cm}{
			$$\Lambda_a(t) =
  			\begin{cases}
    			1 - \frac{|t|}{a}& \mbox{f"ur } |t| < a\\
    			0 & \mbox{f"ur } |t| \geq a\\
  			\end{cases}$$
			$$\mathcal{F:}\quad \Lambda_a(t) \; \laplace \;
			a\left(\frac{\sin(\frac{a\omega}{2})}{\frac{a\omega}{2}}\right)^2 = a \sinc^2\left(\frac{a\omega}{2}\right)$$ \\
		}
		\\ \hline & \\

		\textbf{Sincfunktion \skript{18}} &
		\matlab{sinc}
		\\
		\parbox{6cm}{
			\includegraphics[width=5.5cm]{./bilder/sinc.png}
		}
		&
		\parbox{12cm}{
			$$ \sinc_{\alpha}(t) = \frac{\sin(\alpha t)}{t} \qquad 
			\sinc(\alpha t) = \frac{\sin(\alpha t)}{\alpha t}$$
			$$\mathcal{F:}\quad  \sinc_{\alpha}(t) = \frac{\sin(\alpha t)}{t} \; \laplace \;
			\pi p_{\alpha}(\omega)$$
			$$\mathcal{F:}\quad  \sinc(\alpha t) = \frac{\sin(\alpha t)}{\alpha t} \; \laplace \;
			\frac{\pi}{\alpha} p_{\alpha}(\omega)$$ } \\
		\end{tabular}

		\begin{tabular}{ll}
		\textbf{Impulsfunktion \skript{19}} &
		Diracimpuls, Diracstoss, Deltaimpuls \matlab{dirac}
		\\
		\parbox{5cm}{
			\includegraphics[width=5cm]{./bilder/dirac.png}
		}
		&
		\parbox{13cm}{
			$$\delta(t) =
  			\begin{cases}
    			\infty & \mbox{f"ur } t = 0\\
    			0 & \mbox{sonst}\\
  			\end{cases}$$
		\begin{tabular}{|r|c|l|} \hline
	 		1. & $\delta(at) = \frac{1}{|a|}\delta(t)$ & Skalierung\index{Skalierung}\\ \hline
	 		2. & $\delta(\frac{t-t_0}{a}) = |a|\cdot\delta(t-t_0)$ & Skalierung und Verschiebung  \\ \hline
	 		3. & $\delta(-t+t_0) = \delta(t-t_0)$ & symmetrisch\\ \hline
	 		4. & $\delta(-t) = \delta(t)$ & $\delta(t)=\mbox{ gerade Funktion}$ \\ \hline
	 		\textbf{5.} & $\int\limits_{-\infty}^{\infty}\delta(t-t_0)f(t)dt = f(t_0)$ & \textbf{Siebungseigenschaft}\\ \hline
	 		6. & $\delta(t-t_0)f(t) = f(t_0)\delta(t-t_0)$ &  Abtastung\index{Abtastung}\\ \hline
	 		\textbf{7.} & $\int\limits_{-\infty}^{\infty}A\cdot\delta(t)dt = A$ & \textbf{Spezialfall der Siebungseigenschaft} \\ \hline
	 		8. & $\delta(t-t_0)\ast f(t) = f(t-t_0)$ & Faltung\\ \hline
	 		9. & $\delta(t-t_1)\ast\delta(t-t_2) = \delta(t-t_1-t_2)$ & Faltung\index{Faltung}\\ \hline
			10. & $\delta(t)=\frac{\partial u(t)}{\partial t}$ & Ableitung des Einheitssprungs\index{Ableitung}\\ \hline
			11. & $\delta(t)=\lim\limits_{\omega\rightarrow \infty}\frac{\sin(\omega t)}{\pi t}$ & Definition\\ \hline 
			12. & $\mathcal{L}, \mathcal{F}:\quad \delta(t) \; \laplace \; 1$ 
			& Frequenzbereich
		\\ \hline
		\end{tabular}
		\\
		\vspace{.1cm}\\
		}
		\end{tabular}


%	\subsection{Eigenschaften unterschiedlicher Schwingungsformen}
%
%		\begin{center}
%		\rotatebox{90}{
%		\begin{minipage}[]{24cm}
%			\includegraphics[width=24cm]{./bilder/schwingungsformen.png}\\
%			\textbf{Hinweis zur Rechtecksfunktion:} Für ganzzahlige Tastverhältnisse $\frac{T}{t_{on}} = n$
%			verschwindet die \textbf{n. Harmonische} Schwingung und all deren Vielfache!
%		\end{minipage}}
%		\end{center}	

	\input{idiotenseite/elektrotechnik/subsections/signalformen}	